\documentclass[twoside,12pt]{report}% dvoustranný tisk
%\documentclass[12pt]{report}% jednostranný tisk
% všechny soubory jsou v utf-8. 
	\usepackage{ucs}% pro kódování UTF-8
	\usepackage[utf8x]{inputenc}% kódování vstupních souborů je utf-8

							                  

\usepackage[czech]{babel}% čeština
%\usepackage[slovak]{babel}% slovenština
\usepackage[IL2]{fontenc}% csr fonty (pokud jsou nainstalovány česká postscriptová mísma)
%\usepackage[T1]{fontenc}% EC fonty - háčky a čárky jsou k písmenkům připojovány - nehezké



\usepackage[]{VSKP} % Sablona dle smernice rektora
 % Uvodni desky atd dle smernice rektora
\splithyphens% při rozdělování slov se spojovníkem opakuj spojovník




\begin{document}

\titul% vytiskne titul práce
\abstrakty% vytiskne stránku s abstrakty


\prohlaseni{Prohlašuji, že jsem to všechno opsal a vlastními chybami opatřil.}% prohlášení,
\podekovani{Děkuji všem, kteří mi pomohli okopírovat to, co jsem potom opsal.}% poděkování, nepovinné

% vlastní práce
\obsah% vytiskne obsah

%
%  vlastni text
%
\input{Uvod}% nutné
%
% sem vlastni opsany text, možno vložit více souborů (nejlépe pro každou kapitolu zvláštní soubor)
\input{text}

%
%\input{Zaver}% nutné
\input{Literatura}% nutné
\input{SeznamZkratek}% nutné
\input{SeznamPriloh}% není povinné
\end{document}
